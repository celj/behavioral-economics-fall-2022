% see http://info.semprag.org/basics for a full description of this template
\documentclass[cm,linguex]{glossa}

% possible options:
% [times] for Times font (default if no option is chosen)
% [cm] for Computer Modern font
% [lucida] for Lucida font (not freely available)
% [brill] open type font, freely downloadable for non-commercial use from http://www.brill.com/about/brill-fonts; requires xetex
% [charis] for CharisSIL font, freely downloadable from http://software.sil.org/charis/
% for the Brill an CharisSIL fonts, you have to use the XeLatex typesetting engine (not pdfLatex)
% for headings, tables, captions, etc., Fira Sans is used: https://www.fontsquirrel.com/fonts/fira-sans
% [biblatex] for using biblatex (the default is natbib, do not load the natbib package in this file, it is loaded automatically via the document class glossa.cls)
% [linguex] loads the linguex example package
% !! a note on the use of linguex: in glossed examples, the third line of the example (the translation) needs to be prefixed with \glt. This is to allow a first line with the name of the language and the source of the example. See example (2) in the text for an illustration.
% !! a note on the use of bibtex: for PhD dissertations to typeset correctly in the references list, the Address field needs to contain the city (for US cities in the format "Santa Cruz, CA")

%\addbibresource{sample.bib}
% the above line is for use with biblatex
% replace this by the name of your bib-file (extension .bib is required)
% comment out if you use natbib/bibtex

\let\B\relax %to resolve a conflict in the definition of these commands between xyling and xunicode (the latter called by fontspec, called by charis)
\let\T\relax
\usepackage{xyling} %for trees; the use of xyling with the CharisSIL font produces poor results in the branches. This problem does not arise with the packages qtree or forest.
\usepackage[linguistics]{forest} %for nice trees!
\usepackage{longtable}

\title[Economía de la conducta]{Política conductual de uso de suelo}
% Optional short title inside square brackets, for the running headers.

% \author[Paul \& Vanden Wyngaerd]% short form of the author names for the running header. If no short author is given, no authors print in the headers.
% {%as many authors as you like, each separated by \AND.
%   \spauthor{Waltraud Paul\\
%   \institute{CNRS, CRLAO}\\
%   \small{105, Bd. Raspail, 75005 Paris\\
%   waltraud.paul@ehess.fr}
%   }
%   \AND
%   \spauthor{Guido Vanden Wyngaerd \\
%   \institute{KU Leuven}\\
%   \small{Warmoesberg 26, 1000 Brussel\\
%   guido.vandenwyngaerd@kuleuven.be}
%   }%
% }

\author[Carlos Lezama]{
    \spauthor{Carlos Enrique Lezama Jacinto\\
  \institute{\hfill\break
Instituto Tecnológico\\
Autónomo de México}\\
  \small{\hfill\break
clezamaj@itam.mx}
  }%
  }

\usepackage{natbib}


% tightlist command for lists without linebreak
\providecommand{\tightlist}{%
  \setlength{\itemsep}{0pt}\setlength{\parskip}{0pt}}





\usepackage[spanish]{babel}

\setlength{\parindent}{25pt}
\setlength{\parskip}{\baselineskip}

\begin{document}


\sffamily
\maketitle



\rmfamily

%  Body of the article
Gran parte del esfuerzo realizado en las ciencias del comportamiento es
comprender la base psicológica y biológica de las decisiones subóptimas
(o irracionales) de las personas.

Un objetivo clave de la política de uso de suelo es brindar orientación
sobre políticas al gobierno y a los planificadores con respecto a
cuestiones en geografía, agricultura, conservación ambiental, vivienda,
desarrollo urbano, transporte, entre otras.

El consenso general es que los seres humanos suelen ser víctimas de
sesgos psicológicos cuando se ven abrumados por una gran cantidad de
información, carecen de práctica o experiencia, o se encuentran bajo
presión emocional o cognitiva. Desafortunadamente, el panorama de la
toma de decisiones sobre políticas de uso de suelo está repleto de estos
problemas en tres aspectos principales.

En primer lugar, el proceso de formulación de políticas de uso de suelo
es complejo, con retroalimentaciones excesivamente tardías. La
determinación del uso óptimo de suelo implica la consideración de
impactos a largo plazo en el sistema ecológico, las implicaciones
económicas para los vecindarios y el bienestar general de todos los
residentes involucrados; a menudo lleva años, si no décadas, descubrir
los impactos de una determinada política. Es probable que los
funcionarios gubernamentales y los planificadores utilicen heurísticas
(\emph{mental shortcuts}) al tomar decisiones. Por ejemplo, pueden
centrarse en información que esté disponible de inmediato en lugar de
utilizar el conjunto completo (\emph{availability bias}) o depender
demasiado de las opiniones de expertos (\emph{anchoring effect}). La
falta de retroalimentación inmediata y precisa sobre sus decisiones
puede conducir a resultados no deseados (\emph{over-confidence}).

Segundo, las personas con recursos financieros limitados pueden tener
una capacidad cognitiva obstaculizada por sus problemas
financieros---que superan a otros factores en la toma de decisiones. Por
ejemplo, la presión financiera inmediata que enfrentan los hogares
pobres deja poca capacidad para considerar los beneficios a largo plazo
de sus inmuebles, incluso si el plan está completamente subsidiado por
el gobierno.

En tercer lugar, las personas a menudo están emocionalmente apegadas a
la tierra o las propiedades que poseen (\emph{place attachment}) dada la
intensa interacción con dichos bienes durante un largo período de
tiempo. El efecto de este lazo emocional también se ve intensificado por
el efecto dotación, donde el propietario sobrestima el valor de su
propiedad e impide la adquisición por potenciales residentes.

\citet{Liu} exploran la noción de efecto dotación. Este artículo aborda
un problema importante en el desarrollo rural chino, a saber, los
intentos del gobierno de abordar el problema de los llamados ``pueblos
huecos'' mediante la reubicación de los residentes del pueblo en nuevos
asentamientos, que generalmente consisten en bloques de apartamentos de
gran altura recién construidos.

El término ``pueblos huecos'' se refiere al descuido y el abandono final
de las viviendas rurales cuando la población en edad laboral se va para
buscar empleo en otro lugar, generalmente en las principales ciudades de
rápido crecimiento en el Este de China. Ha habido un gran éxodo de
personas en edad laboral del campo chino en las últimas tres décadas, y
los ancianos y los niños pequeños a menudo son dejados atrás por los
inmigrantes. Alrededor de una quinta parte de las casas rurales se
registraron como desocupadas en una encuesta realizada a mediados de la
década de 2010. Las tierras de cultivo también pueden subutilizarse y
abandonarse en este proceso. Por lo tanto, la reconfiguración del patrón
de asentamiento también es parte de los intentos de utilizar las tierras
agrícolas de manera más productiva mediante la recuperación de tierras
que antes se usaban para viviendas con el fin de agregarlas al área
agrícola productiva.

\citet{Liu} se refieren a la política gubernamental que promueve la
retirada de las viviendas rurales, en virtud de la cual se compensa a
los aldeanos por abandonar sus hogares y mudarse a nuevos bloques de
apartamentos. Su encuesta a aldeanos en Chengdu, provincia de Sichuan,
examina los impactos del efecto dotación y el apego a sus hogares. Estos
últimos son de calidad variable y, por lo tanto, las percepciones de los
aldeanos sobre su valor también varían. Los autores observaron que era
común que existiera una brecha notable entre las valoraciones de la
propiedad por parte de los dueños de casa y el monto de la compensación
que se ofrecía. Las expectativas de compensación de los aldeanos a
menudo superaban el precio de la vivienda comercial local. Esto refleja
un efecto dotación en el que el precio que las personas están dispuestas
a pagar por un bien suele ser menor que el precio que están dispuestas a
aceptar para renunciar al mismo bien.

Los autores también investigaron las expectativas subjetivas de los
aldeanos en relación con sus hogares y cómo las variaciones entre los
aldeanos afectaron las demandas de compensación. Variaciones clave
relacionadas con el apego emocional, el estado de los derechos de
propiedad, los medios de subsistencia, el conocimiento de los derechos
de herencia y el conocimiento de los derechos de hipoteca. En este caso,
el factor clave parece ser la sustituibilidad, donde la sustituibilidad
de un bien se refiere a la importancia de las funciones objetivas del
bien para los individuos. En otras palabras, el bien brinda más apoyo a
las personas y es más difícil de reemplazar, lo que genera una mayor
aversión a las pérdidas y efectos dotación.

En Chengdu, la sustituibilidad se relacionaba principalmente con los
medios de subsistencia de los jefes de hogar. La sustituibilidad de los
medios de vida de los aldeanos afectó positivamente los efectos dotación
en los hogares agrícolas de tiempo completo y afectó negativamente los
efectos dotación en los hogares no agrícolas. Un efecto dotación es
estadísticamente común y relevante en relación con la política que
promueve el retiro de las viviendas rurales.

\bibliography{ref.bib}


\end{document}
